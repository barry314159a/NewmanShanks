\documentclass{article}
\usepackage[normalem]{ulem}
\usepackage{amsthm}
\usepackage{mathtools}
\usepackage{thmtools}
\declaretheoremstyle[headfont=\normalfont]{normalhead}
\usepackage{color}
\usepackage{graphicx}
\usepackage{morefloats}
\usepackage{hyperref}
\usepackage{mathrsfs}
\usepackage{listings}
\hypersetup{
    colorlinks=false,
    linkcolor=cyan,
    filecolor=cyan,      
    urlcolor=cyan,
}
\usepackage{biblatex}
\usepackage{ amssymb }
\usepackage{textcomp}
\usepackage{mathabx }
\usepackage{url}
\usepackage{float}
\usepackage{biblatex}
\addbibresource{main.bib}
\newtheoremstyle{mydef}
{\topsep}{\topsep}%
{}{}%
{\itshape}{}
{\newline}
{%
  \rule{\textwidth}{0.0pt}\\*%
  \thmname{#1}~\thmnumber{#2}\thmnote{\-\ #3}.\\*[-1.5ex]%
  \rule{\textwidth}{0.0pt}}%

\begin{document}
%\theoremstyle{mydef}
\newcommand{\rb}[1]{\raisebox{-.2ex}[0pt]{#1}}
\newtheorem{conjecture}{Conjecture}
\newtheorem{theorem}{Theorem}
\newtheorem*{theorem-non}{Theorem}
\newtheorem{remark}{Remark}
\newtheorem{proposal}{Proposal}
\newtheorem{proposition}{Proposition}
\newtheorem*{proposition-non}{Proposition}
\newtheorem{lemma}{Lemma}
\newtheorem{corollary}{Corollary}
\newtheorem{observation}{Observation}
\newtheorem{definition}{Definition}
\newtheorem*{question-non}{Question}
\newtheorem*{corollary-non}{Corollary}
\author{Barry Brent}

\date{draft 14h 30 May 2023}

\title{Application of the multinomial
theorem to the constant term of a
power series.
}
\maketitle
\begin{abstract}\noindent
We express the constant term of a Laurent series 
in terms of the theory of integer partitions. 
\hskip -.2in
\end{abstract}
\section{Introduction}
We begin with a Laurent series
$f(x) = 1/x+ a_1 +a_2 x + ...$.
Let $C(f^k)$ denote the
constant term of $f(x)^k$.
We study
$C(f^k)$ in this article,
mostly in settings where
(after substituting one of 
several exponential functions for $x$)
$f$ is a meromorphic modular form
for some matrix group.
\newline \newline \noindent
The constant $C(f^k)$ is
a function of the
coefficients 
$a_1, ...,a_k$.
Furthermore the numbers
$C(f), C(f^2), ...$
determine 
$a_1, a_2, ...$.
To see  this, let 
$c_k$ be the coefficient of
$x^k$ in the 
polynomial $h(x) =
(1 +\sum_{n=1}^k a_n x^n)^k$.
It is clear that $c_k = C(f^k)$.
We have
$c_1 = a_1$, 
$c_2 = a_1^2 + 2a_2$, 
$c_3 = a_1^3 + 6a_1 a_2 + 3 a_3$, \it etc. \rm 
We wish to determine the numerical coefficients
in these expressions.
\newline \newline \noindent
Let us write $a_0 = 1$. For an integer partition $\lambda$ of $n$,
let us write \vskip .1in \noindent
$\lambda= (\lambda_1, ..., \lambda_{m(\lambda)})$,
$|\lambda| = n$, $\lambda^* = 
\sum_{t=1}^{m(\lambda)} t \lambda_t$,
and, for $n = 0, 1, ...,k$:
$y_n = a_n x^n$. Finally,
let us write ${n \choose \lambda}$
for the multinomial coefficient
$$
{n \choose \lambda_1, ..., \lambda_{m(\lambda)}} = 
\frac{n!}{\lambda_1! ...  \lambda_{m(\lambda)}!}.
$$
Then $h(x) = 
(\sum_{n=0}^k y_n)^k = $ (by the multinomial theorem)
$$
\sum_{|\lambda| = k} {k \choose \lambda} 
\prod_{t=1}^{m(\lambda)} y_t^{\lambda_t} =
\sum_{|\lambda| = k} {k \choose \lambda} 
\prod_{t=1}^{m(\lambda)} a_t^{\lambda_t} x^{t\lambda_t} =
\sum_{j=0}^{k^2} x^j 
\left (\sum_{\substack{|\lambda| = k\\
\lambda^*= j}
} {k \choose \lambda} \prod_{t=1}^{m(\lambda)} a_t^{\lambda_t}\right ).
$$
Therefore
$$
c_k = \sum_{\substack{|\lambda| = k\\
\lambda^* = k}
} {k \choose \lambda} \prod_{t=1}^{m(\lambda)} 
a_t^{\lambda_t}.
$$
\printbibliography
$\tt{barrybrent@iphouse.com}$
\end{document}